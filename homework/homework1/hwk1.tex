\documentclass[11pt]{article}
\usepackage{fullpage,amsmath}
\usepackage{amssymb,verbatim}
\thispagestyle{empty}

\usepackage{graphicx,amsmath,color,amssymb}
\usepackage{psfrag,picture}

\newcommand{\ba}{\begin{array}}
\newcommand{\ea}{\end{array}}
\newcommand{\be}{\begin{equation}}
\newcommand{\ee}{\end{equation}}
\newcommand{\bd}{\begin{displaymath}}
\newcommand{\ed}{\end{displaymath}}
\newcommand{\bi}{\begin{itemize}}
\newcommand{\ei}{\end{itemize}}
\newcommand{\bn}{\begin{enumerate}}
\newcommand{\en}{\end{enumerate}}
\newcommand{\pa}{\partial}
\newcommand{\f}{\frac}
\newcommand{\ci}{\cite}
\newcommand{\eps}{\epsilon}
\newcommand{\del}{\delta}
%\newcommand{\cal}{\mathcal}
\newtheorem{lem}{Lemma}
\newtheorem{truth}{Theorem}
\newtheorem{prob}{Problem}
\newtheorem{corl}{Corollary}
\newtheorem{rem}{Remark}
\newcommand{\dbl}{[[}
\newcommand{\dbr}{]]}
\newcommand{\dsl}{\{\{}
\newcommand{\dsr}{\}\}} 

\begin{document}
\begin{center}
\textbf{APPM 6610 --- HOMEWORK  \# 1\hfill Due: August 31 in class}
\end{center}

\begin{center}
{\bf Finite Difference Warmup Experiments}
\end{center}

Consider the transport equation $u_t+au_x = 0$ on the 2-periodic domain $-1 \le x \le 1$ with initial data $u(0,x) = e^{-36x^2}$.

Let $u_i^n$ be a gridfunction on the grid $x_i = -1+i h, h = 2/N, i = 1,\ldots,N,$ at time $t_n = kn$, and consider the schemes 
\begin{eqnarray}
u^{n+1}_i &=& u^n_i - ak D_+ u^n_i, \label{up}\\
u^{n+1}_i &=& u^n_i - ak D_- u^n_i, \label{down} \\
u^{n+1}_i &=& u^n_i - ak D_0 u^n_i, \label{cent} \\
u^{n+1}_i &=& \frac{u^n_{i+1}+u^n_{i-1}}{2} - ak D_0 u^n_i, \label{lxf}
\end{eqnarray}
for $i = 1,\ldots,N$ and complemented with periodic boundary conditions $u^n_0 = u^n_N$ and $u^n_{N+1} = u^n_1$. Here    
\begin{eqnarray*}
D_+ w_{i} &=& \frac{w_{i+1}-w_{i}}{h}, \\
D_- w_{i} &=& \frac{w_{i}-w_{i-1}}{h}, \\
D_0 w_{i} &=& \frac{w_{i+1}-w_{i-1}}{2h}.
\end{eqnarray*}
are the standard finite difference operators. 

\begin{enumerate}
\item Take $a = 1$, $N=50$ and set $k \approx 0.5h $ and evolve the solution until time $t = 2$ for all of the schemes above. Hand in plots of the numerical solutions at the final time and the initial data. Which of the schemes appear to be stable?  
\item Repeat but with $k \approx 0.5h^2 $. Hand in plots of the numerical solutions at the final time and the initial data.Now, which of the schemes appear to be stable?  
\item  Next consider a ``random choice'' method. Precisely, when you are computing $u^{n+1}_i$ then (for each $i$) select, at random from a uniform distribution, the right hand side to be either (\ref{up}) or (\ref{down}). Use the stable scheme 75\% of the time (on average). Use $k\approx 0.5h$. Is this scheme stable? 
\item For the above scheme and for a scheme where you randomly (50\% / 50\%) mix the stable scheme of (\ref{up}) and (\ref{down}) with (\ref{cent}) compute the maximum error at time $t=2$ (the exact solution is the initial data). In a log-log plot display the errors as a function of $h$ for $n = 10,11,\ldots, 1000$. In the same figure also plot $h^1$ and $h^2$. Note that (\ref{up}), (\ref{down}) and (\ref{lxf}) are first order in space and (\ref{cent}) is second order accurate in space and all of the schemes are first order accurate in time. What gives?   
\end{enumerate}

\end{document}



