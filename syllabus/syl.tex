% This is a LaTeX file

\documentclass[10pt]{article}
\usepackage{url}
%*********************************************************************
%These are normal margin settings.
\oddsidemargin=0in             % 1in margins at left and right
\evensidemargin=0in
\textwidth=6.5in               % US paper is 8.5in wide

\headheight=0pt                % 1in margins at top and bottom
\headsep=0pt
\topmargin=0in
\textheight=9.0in              % US paper is 11.0in high

\marginparwidth=0.5in
\addtolength{\topmargin}{-0.5in}
\addtolength{\textheight}{1.0in}
\addtolength{\textwidth}{0.5in}
\addtolength{\oddsidemargin}{-.25in}
%*********************************************************************

\begin{document}
\baselineskip=12pt
\pagestyle{empty}
\newcommand{\ds}{\displaystyle}
\newcommand{\be}{\begin{enumerate}}
\newcommand{\ee}{\end{enumerate}}


\centerline{\bf APPM 6610 \hfill INTRO TO NUMERICAL PDES
                          \hfill Fall 2018}
\bigskip

\begin{description}
\item[Instructor:] Daniel Appel\"{o}, daniel.appelo@colorado.edu
\item[Office Hours:] By appointment. \\
                   
                   % 

\item[Lectures:] MWF from 11:00--11:50 AM in Newton Lab

\item[Texts:] Optional: 
\begin{itemize}
\item Theory and Practice of Finite Elements, Ern and Guermond, Springer 2004. 
\item Finite Difference Schemes and Partial Differential Equations, Second Edition, Strikwerda, SIAM, 2004.
\item Time-Dependent Problems and Difference Methods, 2nd Edition, Gustafsson, Kreiss, Oliger, Wiley, 2013.
\item Nodal Discontinuous Galerin Methods, Algorithms, Analysis, and Applications, Hesthaven and Warburton, Springer.
\item More...? 
\end{itemize}
\item[Important Dates:] ~~ \\
     \begin{tabular}{l l l}
     Aug.  27  &  Monday    & Classes begin \\
     Sep.  3  &  Monday  & Labor day holiday, University closed \\
%     Sep.  26  & Tuesday & Mid-term 1, 19.00-20.30, location BIOT A115\\
%     Oct.  30  &  Tuesday  & Mid-term 2, 19.00-20.30, location BIOT A115\\
     Nov. 19-23  &  Mon. - Fri. & Fall and Thanksgiving break\\
     Dec. 13  &  Thursday & Last day of classes \\
%     Dec. 19  &  Wednesday & Final exam, 13.30-16.00 \\
     \end{tabular}

%\item[Grading Distribution:] ~~\\
 %    1/2 Homework \\
 %    1/4 Midterms \\
  %   1/4 Final

%\item [Exams:]
%There will be two midterms and a final. Let me know before Sep. 8 if you will have a conflict with either of the Mid-term exams. 

\item [Homework Policy:]
Homework will be assigned on a regular basis. You are encouraged to work together on the assignments,
however, the major portion should be done on your own. In all cases your submission should demonstrate that {\it you} 
understand the problem and its solution.  

%\item [Course outline:] Given time, we will cover the following topics: \\
%Root finding \\
%Interpolation (by polynomials) \\
%Numerical Differentiation \\
%Integration (quadrature) \\
%ODE's (initial value problems) \\
%Linear algebra \\

\end{description}

\end{document}

